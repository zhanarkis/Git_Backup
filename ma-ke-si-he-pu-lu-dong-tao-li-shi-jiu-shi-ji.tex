\documentclass[DIV=12,%
               BCOR=0mm,%
               headinclude=false,%
               footinclude=false,open=any,%
               fontsize=10pt,%
               oneside,%
               paper=210mm:11in]%
               {scrbook}






\usepackage[noautomatic]{imakeidx}



\usepackage{microtype}
\usepackage{graphicx}
\usepackage{alltt}
\usepackage{verbatim}
\usepackage[shortlabels]{enumitem}
\usepackage{tabularx}
\usepackage[normalem]{ulem}
\def\hsout{\bgroup \ULdepth=-.55ex \ULset}
% https://tex.stackexchange.com/questions/22410/strikethrough-in-section-title
% Unclear if \protect \hsout is needed. Doesn't looks so
\DeclareRobustCommand{\sout}[1]{\texorpdfstring{\hsout{#1}}{#1}}
\usepackage{wrapfig}

% avoid breakage on multiple <br><br> and avoid the next [] to be eaten
\newcommand*{\forcelinebreak}{\strut\\*{}}

\newcommand*{\hairline}{%
  \bigskip%
  \noindent \hrulefill%
  \bigskip%
}

% reverse indentation for biblio and play

\newenvironment*{amusebiblio}{
  \leftskip=\parindent
  \parindent=-\parindent
  \smallskip
  \indent
}{\smallskip}

\newenvironment*{amuseplay}{
  \leftskip=\parindent
  \parindent=-\parindent
  \smallskip
  \indent
}{\smallskip}

\newcommand*{\Slash}{\slash\hspace{0pt}}


% http://tex.stackexchange.com/questions/3033/forcing-linebreaks-in-url
\PassOptionsToPackage{hyphens}{url}\usepackage[hyperfootnotes=false,hidelinks,breaklinks=true]{hyperref}
\usepackage{bookmark}

\usepackage[chinese,shorthands=off,provide*=*]{babel}
\babelfont{rm}[]{Source Han Serif SC}
\babelfont{tt}[Scale=MatchLowercase,%
 Path=/usr/share/fonts/truetype/cmu/,%
 BoldFont=cmuntb.ttf,%
 BoldItalicFont=cmuntx.ttf,%
 ItalicFont=cmunit.ttf]{cmuntt.ttf}
\babelfont{sf}[Scale=MatchLowercase]{Source Han Sans SC}
\usepackage{xeCJK}
\setCJKmainfont{Source Han Serif SC}[]
\setCJKmonofont{cmuntt.ttf}[Path=/usr/share/fonts/truetype/cmu/,%
 BoldFont=cmuntb.ttf,%
 BoldItalicFont=cmuntx.ttf,%
 ItalicFont=cmunit.ttf]
\setCJKsansfont{Source Han Sans SC}[]



\renewcommand*{\partpagestyle}{empty}


% global style

\pagestyle{plain}




\usepackage{indentfirst}

% remove the numbering
\setcounter{secnumdepth}{-2}

% remove labels from the captions
\renewcommand*{\captionformat}{}
\renewcommand*{\figureformat}{}
\renewcommand*{\tableformat}{}
\KOMAoption{captions}{belowfigure,nooneline}
\addtokomafont{caption}{\centering}







\deffootnote[3em]{0em}{3em}{\textsuperscript{\thefootnotemark}~}


\addtokomafont{disposition}{\rmfamily}
\addtokomafont{descriptionlabel}{\rmfamily}


\frenchspacing
% avoid vertical glue
\raggedbottom

% this will generate overfull boxes, so we need to set a tolerance
% \pretolerance=1000
% pretolerance is what is accepted for a paragraph without
% hyphenation, so it makes sense to be strict here and let the user
% accept tweak the tolerance instead.
\tolerance=200
% Additional tolerance for bad paragraphs only
\setlength{\emergencystretch}{30pt}

% (try to) forbid widows/orphans
\clubpenalty=10000
\widowpenalty=10000





% given that we said footinclude=false, this should be safe
\setlength{\footskip}{2\baselineskip}
\setlength{\parindent}{15pt}

\title{翻译\textbar{}马克思和蒲鲁东逃离十九世纪}
\date{}
\author{Hakim Bey}
\subtitle{}






% https://groups.google.com/d/topic/comp.text.tex/6fYmcVMbSbQ/discussion
\hypersetup{%
pdfencoding=auto,
pdftitle={翻译\textbar{}马克思和蒲鲁东逃离十九世纪},%
pdfauthor={Hakim Bey},%
pdfsubject={},%
pdfkeywords={}%
}


\begin{document}









  \begin{titlepage}


  \strut\vskip 2em
  \begin{center}

  {\usekomafont{title}{\huge 翻译\textbar{}马克思和蒲鲁东逃离十九世纪\par}}%
  \vskip 1em
  
  \vskip 2em
  
  {\usekomafont{author}{Hakim Bey\par}}%
  
  \vskip 1.5em



   
   \vfill



  
    \strut\par
  


  \end{center}


     \end{titlepage}
     
     \cleardoublepage







作者:Hakim Bey


上一章:https:\Slash{}\Slash{}nightfall.buzz\Slash{}library\Slash{}fu-li-ye-huo-zhe-shuo-wu-tuo-bang-shi-xue


(致 Mark Sullivan)


人们曾经认为历史是一场我们试图从中醒来的噩梦。但是现在网络资本家说历史确实走到了尽头;想必我们现在都醒了。就像一些糟糕的超现实主义小说中的人物一样,我们从恐惧中醒来,进入一个纯白的世界,却发现自己被困在另一个噩梦的领域。谁会想到“历史的终结”有自己的长靴?


从某种意义上说,20世纪只是19世纪的重演:——同样的工业肮脏、殖民帝国主义、商品化、异化、物质世界的蹂躏、金钱统治、阶级战争等等。 19世纪融化并合并为两个对立的阵营,“民主”和“共产主义”,对革命伟大理想的腐败漫画。 20 世纪仅仅是这两种 19 世纪思想之间的斗争。一方面是资本,另一方面是社会:——泰坦尼克号现代主义的潘趣和朱迪秀——“景观”。


“历史”被认为是这两种力量之间的斗争,无论是在摩尼教的意义上,还是在辩证唯物主义的意义上。如此自然地,当资本在 1989-91 年战胜了一个可悲的后斯大林主义官僚机构时,除了少数垂死的余烬外,其余的都被扑灭,并以廉价的工作地块收购了其余的,社会结束了。世界左翼——它已经根据苏联(支持或反对)来定义自己——崩溃了。当“墙”倒塌时,意识形态右翼(自 1945 年以来一团糟)也失去了焦点。不管我们喜不喜欢,我们已经“超越了左右”。只有资本仍然存在——但资本不是“历史”。资本超越历史,历史因此停滞不前。


现在,我们都准备好迎接 19 世纪的第三次重演——事实上,这是一个理想的 19 世纪(“第三次是一种魅力”),正如 19 世纪第一个伟大的银行家和工业家所设想的那样:——资本的胜利,无人反对,几乎是神圣的。甚至不再需要资本——“主义”——不再需要意识形态!——只是金钱,纯粹而简单——作为永动机的“自由市场”。


或者更好的是:——自 1991 年以来,超过 90\% 的现有货币已经进入了某种 Cyber​​Gnostic 天堂或数字球——因此,货币与生产无关,不受政府控制,而且几乎从未以现金的形式出现。所有国家和大多数个人都“欠”了这个在本质上几乎完全是“精神”但在世界上无所不能的实体。这笔钱做了上帝永远做不到的事:历史在纯粹的存在、形而上学和形而上学的领域中找到了超越市场力量(这仅仅是神性的顿悟表现)的“绝对”;狂喜静滞——历史的终结不是空虚而是圆满,不是止息而是目的。


诚然,左派也梦想着历史的终结,因为历史是一个占有与分离的故事。 右派也梦想着它:作为一个转折点。 第二次来临,卡利年代的终结,乌托邦,浪漫的反应\dots{}\dots{}似乎几乎每个人都希望历史结束,尽管我们显然不同意天堂的定义。 在资本的末世论中,天堂是为极少数人保留的——所以后历史的永恒只能被我们其他人视为地狱。


达到共识-现实状态的幻觉仍然是幻觉。这个断言代表了我们的“诺斯替主义”品牌:——我们意识到“另一个现实”的呼唤,它被淹没了,几乎消失了,只有在极少数和部分情况下才能接近。现在这里的“幻觉”正是作为资本的历史终结,而掩盖这种幻觉的修辞(或神学)由“全球新自由主义”构成,是“全球新自由主义”反人类含义的一种终极表达。政治经济学”-Cyber​​Spencerian 为 CEO 和银行行长鼓吹“适者生存”,用一些加密法西斯主义的“社会计划”概念欺骗,如“安全”、媒体饱和、经济纪律、区域无产阶级化等。在目的和实现的幻觉(由为利润率辩护的人所宣传的底线)背后隐藏着一个更深刻的现实:资本现在不欠任何部门的“交易”,因为它不再需要支持来对抗社会。也就是说,资本有能力背叛它以前的盟友(例如民主、人文主义、宗教、“普遍的”中产阶级),因为它已经“一劳永逸地”战胜了社会;既然资本能够“承受”出卖,那么当出卖承诺获利时,它就会出卖。


然而,即使在这个鲜明而深刻的“现实”背后,也隐藏着一种意识本身必须证明的表达,因为所有“历史”的残骸都被认为甚至掩盖了它最后模糊的轮廓和记忆:——所有无法“移入”的经验再现”——如果你愿意的话,是真实的作为看不见的领域——抵抗的可能性在其上找到根源的感性基础。换言之,就在“社会之死”的那一刻,显然而恰恰是社会已经重生,就像凤凰神话中一样。我们也许生活在秘密历史中,这与那些已经居住在千年的人相称。


就好像历史在我们周围停止了,但让我们仍然在运动,仍然束缚于我们自己的形成。时间废墟中的老鼠:——再一次,古老的幽灵。


这一次,以某种方式从零开始,19世纪的革命必须采取另一种形式,一种不受左翼失败和功利主义、残酷或腐朽理性主义影响的形式。不言而喻,会有反​​对,资本会成功地使反对变得毫无意义,但仍然会有更多的反对。问题是:——我们会简单地重复 19 世纪前和 19 世纪后继的错误吗?还是我们会第三次做对,打破历史终结的睡美魔咒?资本现在已经准备好并准备好在完美中实现自己,在其新的和绝对的后历史普遍性和统一性中实现自己。全新的变异超现实诺斯替资本已经准备好迎接它的千年帝国——“还有更多!” (正如广告总是说的那样)。我们——我们准备好完全屈服于模拟的狂喜了吗(在我们的镜框里,在池畔的永恒,被顶进,被资本的狂喜所吸引)?——或者我们还有其他建议吗?十九世纪:——第三次?还是我们从重复强迫的恍惚状态中清醒过来?我们不断推迟、失望和怨恨的噩梦中的噩梦? “我们”还没有创造历史——现在他们告诉我们为时已晚。或许问题是:——我们能否颠覆历史?我们有第三次机会吗?


[注:在写完这篇文章并将其要点提交给自由读书俱乐部的无政府主义论坛(纽约市,1996 年 12 月 10 日)之后,我被告知让·鲍德里亚在《终结的幻觉(The Illusion of the End)》中提出了与我相似的观点 . 然而,经过考察,鲍德里亚似乎并不是在提出 19 世纪的无限理论,而是暗示在“模拟”的标志下,历史只是简单地逆转了自身。“这甚至不是历史的终结”,并且 ,“按照这个速度,我们可能很快就会回到神圣罗马帝国。” 鲍德里亚只是屈服于肖朗式的悲观主义。 他嘲笑那些谈论“希望”的人(大概他的意思是 E. Bloch 所说的“革命希望”)。 他提出了一些好的观点,但从本质上讲,他的宇宙忧郁是他如此蔑视的胜利资本的精神沉沦的症状,而不是批判。 最后,他通过抵抗所能提供的只是一个具有讽刺意味的投降:


\emph{“我们对罪恶和不道德毫无抵抗力\dots{}\dots{}为什么不是一个完全腐败的世界社会,一个混乱的帝国,一个新的世界混乱\dots{}\dots{}等等等等?”}


在此基础上,鲍德里亚鄙视过去的乌托邦(甚至作为认识论或“主观”客体的过去):


\emph{“考古拜物教谴责它的物品成为博物馆垃圾\dots{}..它也背叛了一种可疑的怀旧\dots{}..我们必须探索我们走过的道路的所有痕迹,在历史的垃圾箱中扎根,恢复最好的和最坏的妄想将善恶分开。” (我的斜体)}


诚然,在历史的垃圾箱中生根发芽是一种不纯洁的行为,与那种排斥和引诱筋疲力尽的鲍德里亚的网络花花公子美学相去甚远。历史的垃圾是想象的堆肥。]


如果我们确实有第三次机会,那么我们可以首先同意马克思列宁主义有它的机会并把它搞砸了,或多或少正是无政府主义者所预测的那样。这是否意味着我们的“第三次机会”可以或可能符合无政府主义革命的结构?无政府主义从来没有真正的机会(一些辉煌的时刻,但没有成功)——那么我们为什么要希望无政府主义能够为变种资本的邪恶魔咒提供反击,一条走出 19 世纪欢乐屋的出路?鉴于无政府主义在历史上与 19 世纪对技术进步、理性规划和普遍的后启蒙文化的信念相同,我们能相信它吗?如果国家的权力现在已经被简单地归结为资本的警察部队,那么废除政府(所有无政府主义理论的必要条件)是否仍然有意义?国家不会在新自由主义的指导下消亡——毕竟,必须存在一些实体来授权银行向政权放贷、给予企业福利、促进货币兑换(占所有“诺斯替资本”的很大一部分) ”),并规范劳动和消费。政府将在一定程度上为资本调停权力,但国家政治将不再表达这种权力的真正运动。金钱已经比国家更强大了。无政府主义理论能否适应这种新情况?


如果无政府主义在试图将自己塑造为意识形态和权力时失败了,它仍然可以指出其思想在实现中被超越的重要领域:——狩猎\Slash{}采集者的旧石器时代政体,以及新石器时代早期农民,他们避免了 在人类存在的 99\% 的时间跨度中出现了分离和等级制度; 以及反抗分离和等级制度的伟大反传统,在分离和等级制度出现的地方出现并更新自己。 这些领域是连续的,既包括也掩盖了历时\Slash{}共时的二分法; 尽管不谈“本质”,但还是会出现某种类似革命精神或精神的东西——在非专制社会中,它在权威出现之前就推翻了权威——而在分离和等级制度的世界中,它想推翻权威和占有。 在“失败”或“成功”中,它持续存在。


在 19 世纪,无政府主义就是这种“精神”的体现,虽然不是唯一的,或许也不是最重要的。 (这毕竟是一次失败,而且是欧洲\Slash{}美国本土的失败,对世界其他地区影响甚微或根本没有影响。)在 20 世纪,无政府主义对马克思主义的批判助长了“新左派”的出现、情境主义、自治主义和其他反莫斯科形式的左派——但只要苏联存在,它就扭曲和污蔑了从左派重建革命的一切努力,并破坏了每一个“第三条道路”,每一个既不\Slash{}也不,每一个非专制倾向。北美无政府主义运动在 1986 年至 1989 年间开始迅速发展甚至组织起来,但在 1991 年左右突然崩溃,这绝非巧合。与左翼的其他所有人一样,无政府主义者对苏联的突然内爆完全措手不及.毕竟,资本主义本身在 1980 年代(第三世界债务危机、S\&L 危机、垃圾债券危机等)似乎数次处于世界末日的边缘,而苏联帝国仍在扩张(尼加拉瓜、阿富汗)。


当然,新世界秩序很难被称为“右翼”的胜利,因为新自由主义不是保守主义的一种形式,甚至也不是法西斯主义的形式(尽管它利用了法西斯主义的技巧)。全球资本的胜利没有国王或牧师、国家、部落、习俗或权利的位置——不需要宗教或国家——甚至不需要它作为面具的自由民主——不需要野性,不需要农场,为了保护(事实上恰恰相反)——除了金钱之外不需要权威。由于像左派这样的保守主义暗示了某种关于人类的理论,因此《资本论》将自己置于左右之外,因为它超越了人类(因此它与信息技术和生物工程——超人类的市场)密切相关。


再一次——我们现在“超越左右”,不是出于选择,而是出于必然。然而与此同时,自相矛盾的是,我们又陷入了“理想的 19 世纪”,即资本的第三次也是终结的运动。在我们目前的情况下,来自 19 世纪和原始 19 世纪的各种想法和见解可能会再次显得相关。例如,关于货币的“旧”左派观点可能会阐明现在,因为早期的理论家已经将资本视为胜利.在 19 世纪中叶,似乎不存在真正的社会运动来反对资本的过度决定的影响。从某种意义上说,我们再次处于相同的位置,尽管可能更悲伤和更明智——因为我们已经看到社会运动(它成功地推迟了 20 世纪资本的胜利)以它自己的背叛而达到顶峰,以苏联。在这种背叛的最后时刻,已经陷入千年,我们在大约 1844 年再次回来(在螺旋点?)。社会运动阻碍了 1917 年以来资本的真实和注定的发展到1989年,已经消失;就好像,有人愤世嫉俗和讽刺地说,就好像从来没有过一样。 H. G. Wells 的时间机器让我们回到了过去的某个时刻——实际上是一整套过去的时刻,就像一个时间性的重写本。我们被迫重温过去,即使资本正在为最终起飞进入永恒的未来做准备。


好吧,我们可以接受任务。也就是说,我们可以将过去视为我们的领域,并在我们的反对和抵抗中发现任何有用的东西。总的来说,这是激进历史和其他学科的项目。但我们可以采取一种更具表演性的方法来处理我们在历史上的强制疏离——我们可以重新审视特定的时刻、转折点,关键事件——尤其是失败的时刻。我们可以篡改那些关键的序列,尝试根据我们目前对事情应该如何发生的理解来纠正它们。在科幻故事中,这个动作会产生“时间旅行者悖论”——一个人可能通过改变过去来改变现在,或者创造另一个现实。从某种意义上说,我们可以在基于这个隐喻的思想实验中“表演”历史,看看我们是否可以从中恢复任何关于我们当前需求的启示。不仅仅是过去片段的拼凑或融合,而是一段新的虚构历史,一个可能曾经存在的乌托邦——或者可能拥有意想不到的未来——或者确实是一种看不见的存在。


例如,我们可以回到社会运动尚未在马克思和无政府主义者之间分裂的时代。为什么?因为我们想想象一场运动会取得马克思主义的成功,而不会因为威权主义背叛社会;——我们想想象一个导致 20 世纪真正革命的 19 世纪,而不是希特勒和斯大林的灾难,以及资本的胜利。我们想象中的历史将写在另一个宇宙中,在那里马克思本人成为无政府主义者——或者更确切地说,冲突从一开始就从未发生过。如此构建的游戏世界不会(我们希望)构成单纯的消遣——因为从某种意义上说,我们此时此地确实面临着一个从未发生过无政府主义和马克思主义的世界——除了《资本论》最终摆脱的噩梦醒来。我们很可能会从对虚构的沉迷中学到真正的战略教训。毕竟,史学本身在很大程度上已经放弃了记录“真实发生的事情”的主张——在这个程度上,所有的历史都是意识的历史,或者是历史与意识之间的反馈。如果我们不得不面对一个与 19 世纪中叶“革命”所面临的情况相似甚至几乎复制的情况,那么我们当然处于一个理想的位置,可以从前人的错误中吸取教训,以及从他们最持久的见解。


马克思和无政府主义者之间的分裂通常可以追溯到 1870 年代马克思和巴枯宁在国际内部争夺控制权和影响力的斗争,当时意识形态的界限已经明确——但更重要和决定性的是卡尔·马克思和皮埃尔之间的争吵——约瑟夫·蒲鲁东,就在 1848 年起义之前(实际上是在 1846 年)。起义后,马克思主义和无政府主义开始分道扬镳,虽然还没有这样的标签。我们必须回到名字之前,找出名字以后想要定义的东西。事实上,我们必须回到事物本身以任何形式出现之前——回到马克思和蒲鲁东参与同一个斗争的那一刻,也就是模糊地称为社会主义的那一刻——那时还没有分裂的迹象。


我们选择的那一刻开始于一种因乌托邦社会主义的失败而蔓延的沮丧情绪——灾难性的失败,如欧文的新和谐、傅立叶主义的方阵,或对圣西蒙门徒的荒谬但丰富多彩的崇拜(他们一度消失了)到东方寻找女性弥赛亚)。像马克思和蒲鲁东这样的年轻激进分子夸大了他们对乌托邦的批评(尽管他们欠他们的债很重),试图定义一种新的社会主义,更“科学”,并专注于工人阶级。


年轻的马克思发表了一些新闻,但还没有出书。蒲鲁东发表了《什么是私有财产?》 1840 年,在巴黎左翼享有盛誉——巴黎是革命之城。 1844 年,一直到 1848 年的普遍骚动已经开始,巴黎是一个鼓舞人心的地方,马克思实现了从激进民主主义者向社会主义者的转变。或许在这方面他受到蒲鲁东的影响。到 1842 年,马克思已经阅读了他的作品,当时他称其为“穿透式的”[McClellan, 54]。蒲鲁东想了解最近的德国哲学,尤其是黑格尔,但没有任何东西被翻译成法语,蒲鲁东也不懂德语(尽管他通过在这些语言中排字自学了拉丁语、希腊语和希伯来语)。马克思当然是一位杰出的青年黑格尔主义者——有人介绍了他们——他们一起度过了几个漫长的夜晚。许多年之后,在蒲鲁东的讣告中,马克思给人的印象是所有的教义都是他完成的(然而蒲鲁东未能正确理解黑格尔)。然而在 1844 年,马克思 25 岁,蒲鲁东 35 岁,是马克思钦佩的一本书的作者。马克思很可能从蒲鲁东的谈话和蒲鲁东的书中学到了一些东西。


马克思在 1844 年对蒲鲁东的看法出现在马克思自己在那一年写的第一本书《神圣家族》中。在这里,他承诺保护蒲鲁东免受埃德加和布鲁诺·鲍尔以及其他德国“真正的”社会主义者对他的攻击。在晚年,马克思自称对《神圣家族》的某些方面感到尴尬,他可能指的是这些段落来赞美蒲鲁东:


\emph{正如对任何科学的第一次批评必然牵涉到它所反对的科学的前提,蒲鲁东的著作《什么是私有财产?》从政治经济学的角度批判政治经济学。——我们不必再深入到从法律的角度批判法律这本书的法律部分,因为我们的主要兴趣是对政治经济学的批判。——蒲鲁东的著作将因此在科学上被对政治经济学的批评所超越,甚至对蒲鲁东所设想的政治经济学的批评也超过了这一点。这项任务只有通过蒲鲁东本人才成为可能,正如蒲鲁东的批评以重农主义者对商业制度的批评、亚当·斯密对重农主义者的批评、李嘉图对亚当·斯密以及傅立叶和圣西门的作品的批评为前提一样。}


\emph{政治经济学的一切发展都以私有财产为前提。这一基本预设被认为是一个毋庸置疑的事实,无需进一步研究,甚至正如萨伊天真地承认的那样,甚至只是“偶然”提及的事实。现在蒲鲁东对政治经济学的基础——私有财产进行了批判性的审查,实际上是第一次坚决的、无情的、同时也是科学的审查。这是他取得的重大科学进步,是政治经济学的革命性进步,第一次使真正的政治经济学科学成为可能。蒲鲁东的论文《什么是私有财产?》对现代政治经济学的重要性不亚于西耶斯 第三等级是什么?是为了现代政治。如果蒲鲁东没有把更广泛的私有财产形式——例如工资、贸易、价值、价格、货币等等——作为私有财产的形式来把握,例如在《德意志-弗朗索瓦年报》中所做的那样(见 F恩格斯的《政治经济学批判纲要》)却用这些经济前提来反对政治经济学家,这完全符合他上面提到的具有历史意义的立场。}


因此,有时,作为一个例外——当他们攻击某些特定的滥用行为时——政治经济学家强调经济条件的人道外观,但在其他时候,在大多数情况下,他们准确地设想这些条件与人道的明显差异,在他们的 严格的经济意义。 他们在这种矛盾中摇摆不定,完全没有意识到这一点。


现在蒲鲁东一劳永逸地结束了这种意识的缺乏。他认真对待经济状况的人道表象,并与他们不人道的现实进行了尖锐的对抗。他要求这些条件实际上应该是它们在概念上的样子,或者更确切地说,应该放弃它们的概念,建立它们的实际不人道。因此,当他不是像其他经济学家那样部分地把这种或那种私有财产作为经济条件的伪造者提出时,他是一贯的,而是完全和普遍的私有财产。从政治经济学的角度来看,他做到了批评政治经济学所能做到的一切。


想要描述什么是财产的观点?埃德加(鲍威尔)先生自然不谈政治经济学,也不谈这本书的独特性——恰恰是它把私有财产的本质变成了政治经济学和法学的重要问题。对于批判性批评来说,一切都是不言而喻的。蒲鲁东对私有财产的否定并没有什么新鲜事。他只泄露了《批判的批判》隐藏的秘密之一。


“蒲鲁东,”埃德加先生在完成他的特色翻译后立即继续说道,“因此在历史中找到了某种绝对的东西,一个永恒的基础,一个指导人类的神——正义。”


蒲鲁东 1840 年的法文著作并没有站在 1844 年德国发展的立场上。这是蒲鲁东的立场,无数法国作家与他截然相反的立场,因此赋予批判性批评以用一个和一个相同的笔触。此外,要对付历史上的这个绝对者,只需要始终如一地应用蒲鲁东自己提出的法则——通过否定来实现正义。如果蒲鲁东没有走那么远,那只是因为他不幸出生于法国而不是德国。


对埃德加先生来说,蒲鲁东以其历史上的绝对性和对正义的信仰而成为神学派,而作为专业批评神学的批判性批评现在可以抓住他的“宗教观念”来表达自己。


“每个宗教观念的特点是,它在一种对立面最终成为胜利和唯一真理的情况下确立教条。”


我们将看到,宗教批判的批判如何在这样一种情况下确立教条:一个对立的“批判”最终战胜了另一个“群众”,成为唯一的真理。 然而,蒲鲁东在大规模正义中感知到一个绝对者,一个历史之神,犯下了更大的不公正,因为正义的批评明确地为自己保留了这个绝对者,这个上帝在历史中的作用。 蒲鲁东做得更多。 他详细展示了资本的流动是如何产生苦难的。


私有财产作为私有财产,作为财富,不得不维持自己的存在,同时也不得不维持它的对立面——无产阶级的存在。这是矛盾的积极方面——私有财产本身就足够了。另一方面,作为无产阶级的无产阶级被迫废除自己,同时也不得不废除它的有条件的对立面——私有财产,这使它成为无产阶级。它是矛盾的消极面,它的内在不安——私有财产的消解和消解。


有产阶级和无产阶级代表着同样的人类自我异化。但前者在这种自我异化中感到舒适和肯定,知道这种异化是它自己的力量,并在其中拥有人类存在的外表。后者在这种异化中感到自己被毁灭,并在其中看到了自己的无能和非人存在的现实。用黑格尔的话来说,无产阶级对自己的屈辱感到仇恨和愤慨——这种感觉必然是由其人性与其生活状况之间的矛盾所驱使的,一种公开、果断、全面地否定那种性质。


因此,在这种对立面中,财产所有者是保守党,无产者是破坏党。从前者产生维持对立的行动,从后者产生破坏它的行动。


在它的经济运动中,私有财产被驱使走向它自己的解体,但这只是通过一种不依赖于它的发展,它是无意识的,它违背了它的意志而发生,它是由事物的本质所带来的——从而将无产阶级创造为无产阶级,这种精神和肉体的痛苦意识到自己的痛苦,这种非人化意识到自己的非人性化,从而超越了自身。无产阶级通过创造无产阶级来执行私有财产对自己的判决,正如它通过为他人创造财富和为自己创造痛苦来执行雇佣劳动对自己的判决一样。当无产阶级取得胜利时,它并不会因此成为社会的绝对方面,因为它只有通过超越自身和对立面才能取得胜利。然后无产阶级及其决定性的对立面——私有财产——消失了。


\emph{当社会主义作家将这一历史角色归于无产阶级时,并不是像批判的批判所假装认为的那样,因为他们将无产阶级视为神。相反。因为全人类的抽象化,甚至人性的表象,在完全发达的无产阶级中实际上是完全的,因为无产阶级的生活条件把当今社会的一切条件都带入了最不人道的焦点,因为人迷失在无产阶级中,但同时,他赢得了对这种损失的理论上的认识,并被迫切的、明显的和绝对迫切的需要(必然性的实际表达)驱使反抗这种非人道——因此无产阶级能够而且必须解放自己。但如果不超越自己的生活条件,它就无法解放自己。它不能超越自己的生活条件而不超越当前社会的所有非人道条件,这些条件总结在它自己的处境中。它并没有徒劳地经历艰苦但硬化的劳动学校。这是一个无产阶级是什么以及它因此在历史上被迫做什么的问题。它的目标和历史行动是在其自身的生活状况以及当代市民社会的整个组织中不可逆转地明确规定的。 [青年马克思的著作,第 362-368 页]}


在这些后来可能使马克思难堪的段落中,对教条主义的批判和对作为一种绝对的正义的辩护,尤其是蒲鲁东的——(事实上,正如我们将看到的,蒲鲁东后来将马克思本人教条化的任务)。但最令人尴尬的是:——正如马克思概述蒲鲁东关于财产和工人阶级的观点以保护他们免受鲍威尔的批判性批评时,很明显,这些观点也是马克思的观点。当人们后来阅读马克思对这些主题的处理时,很明显这些观点仍然是马克思的观点。也就是说,《神圣家族》最尴尬的地方,不是马克思后文所说的不正确,而是正确。它揭示了马克思在某种意义上是一个蒲鲁东主义者,马克思会开始贬低并实际上否认这种联系。


《什么是私有财产?》是一本精彩的书,由于蒲鲁东独特的风格,它仍然“读得很好”,这可以称为把愤怒的讽刺带到了崇高的极端;以生动的悖论与热情的开放相结合的方式传达思想的新鲜和独创性;最重要的是,它对财产作为盗窃的性质的一个核心而强大的洞察力。


\emph{如果我被要求回答以下问题:什么是奴隶制?我应该一句话回答,是谋杀,我的意思马上就明白了。不需要扩展论证来证明从一个人身上夺走他的思想、意志和个性的力量是一种生死攸关的力量。奴役一个人就是杀了他。那么,为什么要回答另一个问题:什么是财产?我不能同样回答,这是抢劫,肯定会被误解;第二个命题只不过是第一个命题的变换?}


\emph{我承诺讨论我们的政府和我们的机构,财产的重要原则:我有我的权利。我的调查得出的结论可能是错误的:我是对的。我认为最好把我的书的最后一个想法放在第一位:我仍然是对的。}


\emph{这样的作者教导说,财产是一种由职业产生并受法律认可的公民权利;另一位坚持认为,这是一项源自劳动的自然权利——这两种学说虽然看起来完全相反,但却受到鼓励和赞扬。我认为,无论是劳动、职业还是法律,都不能创造财产;它是无因的果:我是可指责的吗?}


\emph{财产就是抢劫!\dots{}\dots{}人类思想的一场革命! 东主和强盗在任何时候都是矛盾的,就像他们指定的生物是敌对的一样! 所有语言都延续了这种对立。 那么,你凭什么权威来攻击普遍同意,向人类撒谎呢? 你是谁,竟要质疑万国和历代的判断?}


\emph{尽管如此,我没有建立任何系统。我要求结束特权、废除奴隶制、权利平等和法治。正义,仅此而已:这是我的论点的阿尔法和欧米茄:我将统治世界的工作留给了其他人。 [什么是财产?,第 37-39 页,passim。]}


正如马克思所指出的,《什么是私有财产?》还没有完全发展——但正如他所说,蒲鲁东已经按照他自己的条件使进一步的发展成为可能。蒲鲁东本人当然花了一生的时间来发展这一思想——在某种意义上马克思也是如此——财产作为资本是剥夺(马克思将其表述为“剩余劳动”)的总和。每个人都觉得对方失败了,马克思将用一整本书来“克服”蒲鲁东——但我们走在了故事的前面。


马克思在 1844 年写了另一部著作,但从未出版,甚至没有完成。当它在 1932 年作为经济和哲学手稿出现时,它在马克思学界引起了巨大轰动。在一些批评家看来,马克思之前的马克思似乎已经被揭示出来,一个与后来的政治和“辩证唯物主义”马克思有很大不同的“青年马克思”。其他批评者不遗余力地否认这种印象,认为“认识论”没有中断将马克思与马克思(或更重要的是,马克思与马克思主义!)分开,手稿中没有任何后来马克思未能重复使用或发展的内容——包括“异化”甚至“人文主义”的概念。这个论点在历史上很重要,因为“青年马克思”的拥护者倾向于非正统的,特别是反斯大林主义的解释,而更单一的马克思的捍卫者往往表现为党内线或某种传统主义者。


在这里,我们更感兴趣的是手稿作为 1844 年的文本,而不是近一个世纪后重新发现的文本。关于马克思对蒲鲁东的解读,它告诉了我们什么?我们发现,马克思已经在进行扩展和修正蒲鲁东基本洞见的工程。例如,他非常合乎逻辑地调整了蒲鲁东关于工资的论点。他说(154)“蒲鲁东应该受到批评和赞赏。”换句话说,手稿包含对蒲鲁东的建设性批评,可以想象蒲鲁东能够理解和利用这种有用的“攻击”(因为他自己的座右铭是 Destruam et aedificabo,“我破坏以建造”)。他可能还欣赏手稿的远见卓识——甚至是诗意。当然,他会批准这样的“无政府主义”声明:


\emph{首先要避免的是再次将“社会”建立为对个人的抽象。个人就是社会存在。因此,他的生活表达——即使它不是立即以与他人一起进行的公共表达的形式出现——也是社会生活的表达和主张。 [年轻的马克思,p。 306]}


蒲鲁东会发现马克思在他对“粗鲁”共产主义的批判中与他的观点一致(马克思所说的基于“嫉妒”,先于尼采)。 马克思继续简要地发展了政治共产主义的思想,或者一方面是“民主的或专制的”,或者另一方面是克服了国家——即社会民主主义、无产阶级专政和“消亡 国家。”


\emph{在这两种形式中,共产主义已经知道自己是人的重新整合或回归自我,是对人的自我异化的克服,但是由于它还没有理解私有财产的积极本质,​​同样很少理解需要的人性,它仍然被私有财产所俘虏和感染。它确实掌握了它的概念,但仍然没有掌握它的本质。}


\emph{(3) 共产主义作为对私有财产的积极克服,作为人的自我异化,因此作为通过人并为人实际占有人的本质;因此,作为人在先前发展的全部财富中完全和有意识地恢复到自身,作为社会的人,即人的恢复。这种作为完整的自然主义的共产主义是人文主义,作为完整的人文主义是自然主义。是人与自然、人与人对立的真正解决;它是存在与本质、客观化与自我肯定、自由与必然、个体与物种之间冲突的真正解决。它是解开的历史之谜,并且知道它自己就是这个解。 [青年马克思,第 303-304 页]}


但即使在这一点上,马克思还没有结束他对共产主义的阐述。共产主义本身似乎是一个需要克服的条件:


\emph{但是,对于社会主义人来说,整个所谓的世界历史只是人通过人类的劳动创造和自然对人的发展,所以人的自我创造、自己的形成过程是有目共睹的、无可争辩的证据的。由于人对自然的本质依赖——人对人作为自然的存在,自然对人作为人的存在——已经变得实际的、感性的和可感知的,关于超越人和自然的外星人的问题(这个问题暗示自然与人的不真实性)在实践中已成为不可能。无神论作为对这种不真实性的否认不再有意义,因为它是对上帝的否定,并通过这种否定断言人的存在。但社会主义本身不再需要这种调解。它从理论上和实践上对人和自然作为本质存在的感官知觉开始。它是人的积极的自我意识,不再通过克服宗教获得,正如现实生活是积极的现实,不再通过克服私有财产,通过共产主义获得。共产主义的立场是对否定的否定,因此,对于下一阶段的历史发展来说,是人的解放和康复的必要的实际阶段。共产主义是近期的必然形式和动力原则,而不是人类发展的目标——人类社会的形式。 [同上,p.第314页]}


很明显,马克思认为国家在这样的未来中没有任何作用,在这一点上,他和蒲鲁东的关系仍然比他们后来的追随者想象的要紧密得多。


\emph{国家越强大,国家的政治性越强,就越不倾向于在国家本身的原则中,即在国家所在的现有社会组织中寻找基础和把握社会弊病的一般原则。 是主动的、自觉的、正式的表达。 政治思想之所以具有政治性,正是因为它发生在政治的范围内。 越尖锐、越旺盛,越是无法理解社会弊病。}


\emph{政治的原则是意志。越是片面、越完善的政治思想,越相信意志的万能,越是对意志的自然和精神限制视而不见,越是无法发现社会弊病的根源。 .}


\emph{总的来说,革命——推翻现有的统治权力,瓦解旧的条件——是一种政治行为。然而,没有革命,社会主义就无法实现。只要它需要推翻和解散,它就需要这种政治行动。但是,在它的组织活动开始的地方,在它自己的目标和精神出现的地方,社会主义在那里抛开了政治外壳。 [同上,第 349-350、357 页]}


这些是手稿的高潮词。马克思似乎暗示革命将是“政治的”,但一旦它掌权,国家就会消失。这将与后来的马克思主义关于无产阶级国家的概念相矛盾,后者只是在后历史的某个模糊而遥远的时代通过逐渐“消亡”而不是突然摆脱政治“外壳”而取得成功。手稿中的马克思在这里很难与他后来的无政府主义崇拜者巴枯宁等人区分开来,巴枯宁甚至称他为“大师”。他对异化的热情分析(这里太复杂了)肯定会影响后来的无政府主义读者——如果他们能够阅读的话。到 1932 年,没有无政府主义者能够在苏联和斯大林存在的历史背景之外阅读手稿。直到 1989-91 年之后,反威权主义者终于可以接近这个文本并结识一位年轻的马克思,他\dots{}\dots{}几乎\dots{}\dots{}一个无政府主义者。


为了理解接下来发生的事情,有必要承认,从各方面来看,马克思和蒲鲁东并不是世界上最容易相处的人。尤其是马克思,他的性格中有着强烈的威权主义倾向,作为一个不容忍任何分歧的人给许多人留下了深刻的印象。他经常将基本动机归咎于他的对手(当然有时是公正的),但可能对此非常不愉快。蒲鲁东为自己塑造了(或在自己身上意识到)一个自称多刺顽固的农民的性格。他是一个古板的道德家(因此他对傅立叶感到着迷和厌恶)并且对妇女和犹太人有荒谬的小资产阶级钩针编织。他是工人阶级,自学成才,对此很敏感。马克思出身资产阶级,受过良好的教育。马克思是一个犹太人——尽管他完全世俗化了,甚至他自己也有点反犹。而且他作为父权制万神殿之一当然也没有逃脱女权主义者的批评(他曾经说过他钦佩女性的“弱点”)——所以也许我们可以在马克思和蒲鲁东的 19 世纪失败和迷信的水平上平分秋色.无论如何,马克思和蒲鲁东注定不是朋友。在另一个宇宙中,他们可能会因为两个激进的脾气暴躁的人联合起来反对私有财产而一拍即合。想象这不仅仅是幻想,因为通宵的黑格尔公牛会议暗示了某种温暖——更重要的是,他们的文本中展示了一种哲学上的相容性——某种和谐。但事实并非如此。


马克思和蒲鲁东之间的争吵是 19 世纪历史的一个关键点,也是当今世界的一个关键点——没有联系、没有综合,影响了社会运动的后续进程,因此资本的也是如此。马克思主义者最终垄断了社会并为它赢得了整个“第二”世界,他们这样做的基础是对蒲鲁东无政府主义的无情仇恨,它是运动中唯一的重要竞争对手。至于无政府主义者,尽管他们可以在很多事情上受到指责,但他们对马克思主义的批判被证明是完美无缺的。没有人能把它从无政府主义中拿走:——它为马克思主义所预言的一切都成真了。资本主义从来没有理解马克思主义,也没有发展出一种能够将其作为理论来反对它的理论。然而,无政府主义这样做了。悲剧的讽刺。


1845 年,当马克思离开巴黎时(被坏账追赶,这是一种常见的模式),他相信自己仍然与蒲鲁东保持着良好的关系。第二年,住在布鲁塞尔的马克思构想了成立一个通讯委员会的想法,将欧洲的社会主义者连接到一个信息交换网络中(我想现在它是一个网页)。他决定邀请蒲鲁东加入。他们的书信往来是两人之间唯一的通信,对于理解争吵至关重要,因此我们必须将两篇文本都包括在内。


\emph{马克思致蒲鲁东}


\emph{Brussels, May 5, 1846}


\emph{我亲爱的蒲鲁东}


\emph{自从我离开巴黎以来,我经常想给你写信,但迄今为止,与我的意愿无关的情况阻止了我这样做。让我向你保证,我保持沉默的唯一原因是工作不堪重负,而且因为搬家等麻烦而忙得不可开交。}


\emph{现在让我们跳入媒体资源!我和我的两个朋友,弗雷德里克·恩格斯和菲利普·吉戈特(都在布鲁塞尔),组织了与德国共产主义者和社会主义者的持续通信,既要讨论科学问题,又要监督流行出版物作为社会主义宣传,可以通过这种方式在德国进行。然而,我们通信的主要目的是让德国社会党人与法国和英国社会党人接触。让外国人了解即将在德国发生的社会主义运动,并向在德国的德国人通报法国和英国的社会主义进展情况。这样,就有可能表达意见分歧。随之而来的是思想交流和公正的批评。这是社会运动在其文学表达中应该采取的步骤,以摆脱其民族局限。在行动的时候,对国内外的情况有所了解,对每个人来说无疑是大有裨益的。}


\emph{除了德国的共产党人,我们的信件还包括巴黎和伦敦的德国社会主义者。我们与英国的联系已经建立;至于法国,我们都认为在法国找不到比你更好的通讯员了。如您所知,迄今为止,英国人和德国人对您的欣赏超过了您的同胞。}


\emph{所以你看,这只是一个定期通信的问题,并确保它有能力跟踪各国的社会运动,一个让它变得有趣、丰富和多样化的问题,这是一个人的工作永远做不到的 达到。}


\emph{如果您接受我们的建议,我们寄给您和您寄给我们的信件的邮费将在这里支付,在德国筹集的资金将用于支付通信费用。}


\emph{我们希望您写信的地址是 M. Philippe Gigot, 8, rue Bodenbrock。他也是签署布鲁塞尔信件的人。}


\emph{我不需要补充说,您必须对整个通信保持绝对保密;在德国,我们的朋友必须谨慎行事,以免损害自己。}


\emph{给我们一个早期的答复,相信真诚的友谊}


\emph{鄙人}


\emph{卡尔马克思}


\emph{Brussels, May 5, 1846}


\emph{P.S.} \emph{让我在这里谴责 M. Grün 在巴黎。这个人不过是一个文学骗子,一个江湖骗子,想搞现代思想。他试图用夸夸其谈、傲慢的词句来掩饰自己的无知,但他只是成功地通过他的自大胡说八道让自己看起来很可笑。而且,这个人很危险。他滥用了他的无礼使他与知名作家建立的关系,将他们用作梯子,并在德国公众的眼中妥协。在他关于法国社会主义者的书中,他敢于称自己为蒲鲁东的导师(Privatdozent,德国的学术等级),声称向他揭示了德国知识的重要原则,并拿他的著作开玩笑。因此要小心这种寄生虫。也许稍后我会再次向您提及这个人。}


\emph{我很高兴有机会告诉你我很高兴与像你这样杰出的人建立关系。同时,请允许我自己签名,}


\emph{鄙人}


\emph{Phillippe Gigot}


\emph{就我而言,我只能希望您会批准我们刚刚向您提出的项目,并且您将有足够的义务不拒绝我们的合作。我可以说你的作品让我对你怀有最大的敬意}


\emph{鄙人}


\emph{弗雷德里克·恩格斯}


\emph{蒲鲁东致马克思}


\emph{Lyon, May 17, 1846}


\emph{我亲爱的马克思先生}


\emph{我心甘情愿地同意成为您通信的阶段之一,其目标和组织似乎是最有用的。然而,我不打算写长篇或经常给你写信,因为我的各种职业以及我天生的懒惰使我无法做出这些书信的努力。我还将冒昧地对你信中的不同段落提出的若干保留意见。}


\emph{首先,虽然我在组织和实现问题上的想法目前已经很确定了,但至少在原则上,我相信我有责任,而且所有社会主义者都有责任,时间还持批评和怀疑的态度。简而言之,我向公众承认,经济学几乎完全是反教条主义的。}


\emph{无论如何,让我们一起努力去发现社会的规律,这些规律的实现方式以及我们能够发现它们的过程。但是,看在上帝的份上,当我们推翻了所有先验的教条后,不要让我们想到轮到我们灌输人民。不要让我们陷入你的同胞马丁路德的矛盾中。他一推翻天主教神学,就立即着手建立新教神学,不断诉诸绝罚和诅咒。三百年来,德国的全部关切一直是摧毁路德的大杂烩。让我们有一个好的和诚实的辩论。让我们为世界树立一个明智和有远见的宽容的榜样,但不要仅仅因为我们是运动的领导者,就不要煽动新的不宽容。让我们不要把自己定位为一种新宗教的使徒,即使它是逻辑或理性的宗教。让我们欢迎和鼓励所有的抗议,让我们摆脱所有的排他性和所有的神秘主义。让我们永远不要认为任何问题已经用尽了,当我们用完最后一个论点后,让我们重新开始,如果有必要,用雄辩和讽刺的方式重新开始。在这种情况下,我很乐意加入你的协会,否则我不会。}


\emph{我还必须对你信中的短语“在采取行动的时候”提出一些意见。或许你仍然认为,如果没有帮助的政变,没有过去所谓的革命,但实际上只是一次震动,就不可能进行改革。我承认,我最近的研究使我放弃了这个观点,我理解并愿意讨论,因为很长一段时间我自己都持有它,我认为这不是我们成功所需要的,因此我们不得将革命行动作为社会改革的手段,因为这种所谓的手段只是诉诸武力和武断。简而言之,这将是一个矛盾。我是这样提出问题的:我们怎样才能通过某种经济学体系,将被另一种经济学体系从社会中抽走的财富重新放回社会?换句话说,通过政治经济学,我们必须将财产理论与财产理论相矛盾,从而创造出你们德国社会主义者所谓的共同体,而目前我只会称之为自由或平等。现在我想我知道可以很快解决这个问题的方法了。因此,我宁愿一点一点地烧毁财产,也不愿通过为财产所有者举办圣巴塞洛缪日来赋予它新的力量。}


\emph{我的下一部作品,目前正在印刷中,将进一步向您解释这一点。}


\emph{我亲爱的哲学家,这就是我现在的立场。我可能弄错了,如果发生这种情况并且你给我手杖,我会在等待报复的同时愉快地忍受它。我必须顺便补充一点,这似乎也是法国工人阶级的感受。我们的无产者如此渴求知识,如果我们能给他们喝的只是血,他们会非常不喜欢我们。简而言之,在我看来,使用灭绝语言是非常糟糕的政策。严厉的措施会适得其反;在这一点上,人们不需要劝告。对于德国社会主义中似乎已经存在的微小分歧,我真诚地感到遗憾,您对格鲁恩先生的投诉就证明了这一点。我很担心您可能以错误的眼光看待这位作家,我呼吁马克思先生,请您做出平衡的判断。 Grün 在流亡中没有财产,有妻子和两个孩子,除了笔外没有其他收入来源。除了现代思想,他还能利用什么来谋生?我理解你的哲学愤怒,我同意人类的圣旨永远不应该被用作讨价还价的柜台。但在这种情况下,我必须考虑不幸,极端的必要性,我原谅这个人。啊,是的,如果我们都是百万富翁,情况会好得多。我们都应该是圣徒和天使。但是我们必须活着,而且你知道这个词与纯粹联想理论所表达的意思相去甚远。我们必须生活,也就是说,买面包、燃料、肉,我们必须付房租。天哪!一个兜售社会观念的人,其功德不亚于兜售布道的人。我对格伦自称是我的导师一无所知。辅导什么?我只关心政治经济学,他几乎一无所知。我把文学看成是小女孩的玩物,至于我的哲学,我对它了解得够多,偶尔也能轻描淡写。 Grün 根本没有向我透露任何信息,如果他声称这样做了,那他就是自以为是,我相信他会后悔的。}


\emph{但我所知道的,并且比我谴责对虚荣的轻微攻击更重视的是,由于我对您的著作,我亲爱的马克思先生以及那些M. Engels 以及费尔巴哈非常重要的著作。应我的要求,这些先生们已经很好地用法语(不幸的是我完全看不懂德语)为我分析了最重要的社会主义出版物。在他们的恳求下,我必须在我的下一部作品中提及一些女士们的作品(无论如何我都会自愿这样做)。马克思、恩格斯、费尔巴哈等。 . .最后,格伦和尤尔贝克正在努力使巴黎德国殖民地的神圣火焰保持活力,而咨询他们的工人对这些先生的尊重在我看来似乎是他们意图诚实的可靠保证。}


\emph{亲爱的马克思先生,我很高兴看到您扭转因一时的恼怒而做出的判断,因为当您给我写信时,您正处于愤怒的心态。 Grün 告诉我他希望翻译我现在的书。 我意识到这个翻译比任何其他翻译对他都更有帮助。 因此,如果您和您的朋友能在这个场合帮助他出售一件在您的帮助下无疑会使他受益多于我的作品,我将不胜感激,不是出于我自己的利益,而是出于他的利益。}


\emph{亲爱的马克思先生,如果您向我保证您的帮助,我将立即将我的证明寄给格伦先生。我认为,尽管您有个人不满,我不打算对此作出判断,但这一行动方针将是各方的功劳。}


\emph{鄙人}


\emph{向您的朋友 Engels 和 Gigor 先生致以最诚挚的问候。}


\emph{[蒲鲁东,文选,p。 147-154]}


乍一看,这些信件似乎构成了进一步通信的合理基础。蒲鲁东在某些条件下同意加入委员会——但显然这些条件无法满足,因为马克思从未回复过——除了他对蒲鲁东的下一本书《贫困哲学》的严厉抨击。马克思在教条主义问题上已经用与蒲鲁东信中的措辞相似的方式表达了自己的观点,但显然他把蒲鲁东的言论当成了个人。至于关于暴力的分歧,这可能显得更严重。在国际的 1870 年代,马克思及其追随者愿意与避免诉诸选举政治或暴力革命的蒲鲁东主义者合作(他们正在发展总罢工的概念)。当然,马克思最终成功地清除了国际上所有的无政府主义者——但他至少愿意考虑一次统一战线。然而,在 1846 年,他显然不愿与蒲鲁东合作,或许他可以将他对蒲鲁东的“条件”的厌恶解释为原则上拒绝蒲鲁东的非暴力原则。然而,由于马克思切断了对应关系,我们只能假设,不能证明。这个问题似乎没有在后来的辩论中提出——尽管我很可能错过了它,因为我对这篇文章的阅读是肤浅的。


蒲鲁东是有意冒犯马克思吗?或许他不自觉地做到了。但蒲鲁东与马克思一样是 19 世纪的泼墨论战家,不怕树敌。如果他想冒犯他,他本可以这样做,而无需像承诺在他的下一本书中引用马克思这样的自相矛盾的尊重迹象!即便如此,蒲鲁东的信还是有一种略显优越的气质,可以理解马克思为什么生气。


马克思和蒲鲁东后来都参加了 1848 年的革命。蒲鲁东仍然拒绝暴力,但全力支持巴黎的工人阶级,甚至允许自己被选为公职——他后来对此感到遗憾: 一个严重的错误。 马克思在科隆发现自己在为资产阶级革命工作,他认为这是任何成功的无产阶级革命的必要先驱——尽管对于一个自称为共产主义者的人来说,这需要一些奇怪的伙伴。 换言之,1848年马克思和蒲鲁东都对革命的现实妥协了。 理论上,无论如何,他们本可以在 1846 年革命暴力的理论问题上妥协。


他们为什么吵架? 马克思主义者和无政府主义者都相信(如果他们仔细想想的话),最深层次的问题一定是让两位创始人产生了分歧,他们的分歧是深刻的、无法弥合的,命运已经注定了分裂,没有选择。 但我们已经看到(或至少怀疑)哲学鸿沟可能没有 后来想的认为的那么大。 无论如何,如果马克思和蒲鲁东在基本原则上存在分歧,那么这些原则究竟是什么? 如果我们分析 1921 年甚至 1880 年的马克思主义和无政府主义,我们可以发现关于具体事物的明显分歧。 但这在 1846 年也是真的吗?


那年晚些时候,蒲鲁东的下一本书问世了:Système des Contradictions Economiques, ou Philosophie de fa Misère,分两卷(其中只有第一卷被翻译成英文为《苦难的哲学》。我更喜欢称它为《贫困的哲学》为了与马克思的回答《哲学的贫困》直接对比。)蒲鲁东和马克思一样,一直在吞噬古典经济学家,现在他试图从经济范畴的角度重新思考他关于财产的发现。在《什么是财产?它肯定在接缝处分崩离析,尽管它被大量蒲鲁东式的好言辞和悖论所击穿,减轻了政治经济学的负担。


例如——在他对卷的介绍和结论中。我,蒲鲁东以一种非常不寻常的方式对待宗教。他从“上帝的假说”开始,甚至批判庸俗的无神论,最终走到了只能称为无政府主义神学的立场——对上帝的攻击。


\emph{就我而言,我要说:人的首要责任是变得聪明和自由,就是不断地从他的思想和良心中寻找上帝的概念。因为上帝,如果他存在的话,本质上对我们的本性是敌对的,我们根本不依赖他的权威。尽管有他,我们还是达到了知识,尽管有他,但我们还是达到了舒适,尽管有他,我们还是达到了社会;我们提前采取的每一步都是我们粉碎神性的胜利。}


\emph{不要再说上帝的道路是不可逾越的。我们已经深入了这些方式,并且在那里我们从血书中读到了上帝无能的证据,如果不是他的恶意的话。我的理性,长期受辱,逐渐上升到无限的高度;随着时间的推移,它会发现它的缺乏经验所隐藏的一切;随着时间的推移,我将越来越少地成为不幸的工人,通过我将获得的光明,通过我的自由的完善,我将净化自己,理想化我的存在,成为创造之首,与上帝平等.万能者本可以阻止或没有阻止的混乱时刻指责他的天意,表明他缺乏智慧;无知、被遗弃、被背叛的人,在获得荣誉的道路上所取得的最微小的进步,都会给他带来无法估量的荣誉。神还凭什么权利对我说:要圣洁,因为我是圣洁的?说谎的精神,我会回答他,愚蠢的上帝,你的统治结束了;寻找其他受害者的野兽。我知道我不是圣洁的,也永远不可能成为圣洁;如果我像你,你怎么可能是圣洁的?永恒的父亲,朱庇特或耶和华,我们学会了认识你;你是,你曾经是,你将永远是亚当的嫉妒对手,普罗米修斯的暴君。}


\emph{[\dots{}]}


\emph{你的名字,长久以来,学者的遗言,法官的认可,王子的力量,穷人的希望,悔改的罪人的避难所——我说,这个无法传达的名字,从此成为蔑视的对象和诅咒,将成为人类的嘶嘶声。因为上帝是愚蠢和怯懦;上帝是伪善和虚伪;上帝是暴虐和悲惨的;上帝是邪恶的。}


\emph{[\dots{}]}


\emph{宗教中的上帝,政治中的国家,经济中的财产,这就是人类与自己格格不入的三重形式,并没有停止用自己的双手撕裂自己,今天它必须被拒绝。 [苦难哲学,448-457 passim。]}


有趣的是,蒲鲁东年轻时曾考虑过写一本关于基督教异端的书,而且他特别喜欢革命的诺斯替二元论者,如卡特里派(Cathars)。对于诺斯替派来说,宗教的耶和华是一个邪恶的造物主——实际上是撒旦的——而真正未知的上帝完全远离任何物质创造,完全反对一切生成。蒲鲁东挽救了邪恶的上帝的想法,但驱逐了纯粹精神的概念,从而接近纯粹的虚无主义。普鲁东派成为众所周知的无神论者,巴枯宁后来宣称如果上帝存在,我们必须杀死他,这是在呼应蒲鲁东。对教会来说,蒲鲁东是一个反基督者。


蒲鲁东的书出版一年后,马克思进行了报复(有点冷酷),发动了蒲鲁东在信中曾挖苦预言的“鞭笞”——一整本书都致力于企图屠杀,并称之为《哲学的贫困》。马克思完成了建设性的批评;他打算推翻蒲鲁东,并把他加入已经越来越多的成功人物的名单中,如鲍尔兄弟,或他以前的同志年轻的黑格尔派(《德意志意识形态》很好地处理了他们,尽管马克思没有把它印出来),或他以前的社会主义者。马克思已经开始了通过排斥、诽谤甚至歪曲其“敌人”来定义马克思主义的毕生战略——这一战略用谴责、清洗和背叛左翼潜在盟友的气味污染了所有未来的马克思主义。一些马克思主义者认为《哲学的贫困》是一部很好很重要的作品,但是,除了马克思偏离主题的几段,我发现这是一部很难——事实上令人不快——的作品。实际上,我已经费力地读完了蒲鲁东的书,我怀疑这比大多数马克思学者所做的还要多,我不断地被马克思对蒲鲁东的立场的倾斜的、有时是完全扭曲的表述所困扰。例如,他以此开头:


\emph{M(meter的缩写))蒲鲁东的著作不仅仅是一部政治经济学专著,一本普通的书;这是一本圣经。“神秘”、“从上帝的怀抱中攫取的秘密”、“狂欢”——什么都不缺。但是,现在人们比世俗的作家更认真地讨论先知,读者必须辞职,和我们一起走过《创世纪》的枯燥和阴郁的学问,以便以后和蒲鲁东先生一起进入超级社会主义的空灵和富饶的领域。[贫困,第26页]}


在这里,马克思清楚地暗示蒲鲁东是某种宗教社会主义者,由于当时大多数法国社会主义者都是宗教人士,这不会使读者感到惊讶。要么马克思错过了蒲鲁东使用“上帝的假设”的讽刺,要么他简单地跳过了这本书的导言和结论——要么他从事故意曲解。


简而言之,《贫困的哲学》和《哲学的贫困》这两本书应该为我们阐明无政府主义与马克思主义(所谓的深渊)争吵的真正根源——但它们却没有做任何类似的事情。蒲鲁东沉迷于对经济学的一些不清楚的思考,马克思能够纠正他。例如,在蒲鲁东认为他发现了一些原则的地方,马克思能够证明其他一些早期经济学家首先想到了它——因此,马克思必须向读者隐瞒他与蒲鲁东的基本一致,他通过指责抄袭来做到这一点,或者无知,或(马克思主义者最喜欢的)可疑动机。在这里,马克思发展了一种暗杀性格的手段,这种手段将成为左派常年的最爱:——某某声称自己是一名革命者。但是请看:——这位某某关于(比如说)工资平等的想法是不正确的。如果实施这样的想法,它们只会失败,从而帮助资本主义。所以某某不是革命者,是无产阶级的资产阶级叛徒,必须肃清。意图一文不值——党的路线高于个人意志。例如,蒲鲁东是“资产阶级”(第 164 页),甚至更糟的是“小资产阶级”(第 167 页)。别介意蒲鲁东想废除财产(资产者的奇怪行为); “让我们换一种说法:蒲鲁东先生并没有直接说资产阶级的生活对他来说是永恒的真理。他通过神化以思想形式表达资产阶级关系的范畴,间接地说明了这一点。”别介意蒲鲁东出身农民,以工人(排字员)为生——尽管如此,他还是小资产阶级。他对女性和性的态度证明了这一点。


难道这就是无政府主义和共产主义之间的巨大鸿沟吗?不,很明显还有更多的东西。但是人们会徒劳地研究这两部作品来发现它。我不想轻视关于经济学的争论,这在很大程度上是我无法理解的——但似乎很明显,马克思本可以尝试一种建设性的批评(正如他在《神圣家族》和《手稿》中愿意做的那样),而不是一场破坏性的闪电战。关于政治暴力和罢工问题的分歧显然是重要的,但不是敌对的充分理由(毕竟马克思在邀请蒲鲁东参加通信委员会时就已经知道他对这些问题的看法)。简而言之,我们必须在别处寻找无政府主义和共产主义之间的真正分歧。被马克思的突然袭击深深伤害的蒲鲁东,对他的《哲学的贫困》一书作了注解,显然是打算作出正式的回应。然而,他被1848年的事件分散了注意力,再也没有回到这个项目上。对这些注释的研究无疑会进一步我们的调查,但不幸的是不能进行;然而在W Pickles的文章“马克思和蒲鲁东”(1938)中给出了几个例子。从这些可以清楚地看出,这场争论中至少出现了一个最重要的话题:唯物主义历史观的问题。按照马克思(以及后来的马克思主义者)的说法,蒲鲁东不是唯物主义者,但他引入了“宗教”或伦理范畴或“绝对”如“正义”来支持他对历史的解释。然而,马克思主义没有使用绝对性,因为绝对性只不过是表象,而马克思主义关注的是真正的物质性,即“科学”的客观性。然而,蒲鲁东本人在他的马克思著作的一个旁注中认为,即使他似乎是从抽象的原则中推理,他实际上并没有否认唯物主义概念的有效性。“好了,”他说,这是针对马克思在这一点上的一次攻击而言的,“请允许我对你说些什么!我想让这些原则选择代表知识分子,而不是代表事实的原因?”[皮克尔斯,第251页]


皮克尔斯补充说,“毫无疑问,可以从蒲鲁东的作品中引用无数段落,这些段落雄辩地阐述了唯物主义概念,”给出了《贫困的哲学》的几个例子。当然,他接着说,唯物主义对蒲鲁东来说实际上变成了一个道德范畴,所以在某种意义上,马克思的批评不是没有意义的。但是这里我们可以提出一个更有趣的问题。“年轻的马克思”自己不也以同样的方式对待唯物主义,即,作为一个道德范畴吗?回到手稿我们发现这样的段落:


自然的人类本质主要只为社会人而存在,因为只有在这里,自然才是与人的联系,作为他对他人的存在和他们对他的存在,作为人类现实的生命元素——只有在这里,自然才是人类自身的人类存在的基础。只有在这里,人的自然存在才成为他的人的存在,自然才成为人。因此,社会是人与自然的完整的、本质的统一,是自然的真正复活,是人的完满的自然主义和自然的人文主义。[青年马克思,第305—306页]


这样看来,在1844年,马克思和蒲鲁东在唯物史观的问题上并没有真正的分歧。1846年马克思声称存在差异,但蒲鲁东看不到这一点(“我不幸像你一样思考!”).然而,后来,一个真正的差异出现了,或者最终得到了澄清(取决于你对“年轻的”马克思的看法),这种差异导致了马克思主义和无政府主义之间非常明显的分歧,例如,在关于革命“不可避免性”的对立观点中表现出来。在辩证唯物主义看来,革命是不可能的,直到资本本身在某种意义上得到完善,从而屈服于自身的内在矛盾。然而,像古斯塔夫·兰道尔这样的普鲁登无政府主义者认为,革命是对青年马克思所认为的异化的痛苦的一种反应,这种痛苦是真实的,现实的,资本主义也可能是永恒的,因为我们可以从未来的理论危机中获得所有的满足。这个问题在1968年对我这一代人来说达到了顶点,当时法国共产党试图镇压五月的起义。的确,概括了60年代整个计划的巴黎“事件”以失败告终。但是斯大林主义也以失败告终。整个社会运动在1989-91年以失败告终。这篇文章的目的是问这一失败是否与1844年马克思和蒲鲁东之间的争论有关,我们对答案的兴趣不仅仅是学术上的。在某种意义上,我们站在马克思和蒲鲁东在1844年站的地方——所有相同的问题对我们来说又是活生生的。我们开始怀疑他们的理论并非不可调和;也许它们甚至是互补的。我们开始怀疑这场争吵是个大错误。


我们可以通过多种方式检验我们的怀疑。例如,我们可以尝试一个科幻思想实验,问如果马克思和蒲鲁东没有吵架,而是克服了分歧,会发生什么。蒲鲁东会接受马克思关于经济学的观点,并且会掌握黑格尔的辩证法。马克思会放弃教条主义,发展他的异化理论,而不是放弃或扭曲它。在通讯委员会中联合起来,这两位天才本来会作为战友参加 1848 年的战斗,而革命的失败不会像现在这样严重震撼他们。委员会会更早地发展为国际,到 1871 年,国际将能够以蒲鲁东诺-马克思主义的思想支配公社。公社党在巴黎夺取银行,就可以建立一个民主共和国,并直接与普鲁士和平相处。此后欧洲将无情地走向社会革命——首先是德国,然后是俄罗斯、意大利、西班牙等。到 19 世纪末,这个联合起来的欧洲将反对英美在中东、亚洲、非洲和南美洲;因此,所有这些地区都会为社会运动而解放。到 20 世纪中叶,内战或革命将席卷英国和美国,并导致一场内外对资本的最终战争——而地球将为社会赢得胜利。在这一点上(大约在 1999 年左右),我想 UFO 真的会最终到达并邀请 Terra 加入银河联邦,在螺旋星云之外的某个地方与资本主义作斗争\dots{}\dots{}当沉迷于这样的事情时,不可能避免 ad astra,ad absurdum 综合症奇妙的猜测。


事实可能会证明,追踪实际发生的事情比追踪另一个宇宙的分叉路径更有用。马克思主义和无政府主义之间真正的分歧集中在国家问题上——更抽象地说,是权威问题。无论是通过革命还是立法手段,马克思主义者决心夺取政权,尽管他们似乎不清楚这种夺取是资本主义被战胜的条件还是仅仅是资本主义被战胜的标志。然而,无政府主义者主张立即摧毁国家作为社会革命的先决条件,尽管他们不同意这些目的是否可以通过革命战争或革命但和平的经济手段来实现。马克思主义者认为有一些正义的无政府主义者是错误的和无效的;无政府主义者将马克思主义者视为独裁的马基维利式阴谋家,并预言掌权的共产主义将比掌权的资本主义更糟糕。他们是正确的。


尽管马克思的教条主义,无数的马克思主义教派兴起,每个都声称正统的斗篷(包括一个托洛茨基在英格兰的团体,实际上相信不明飞行物和银河联邦!).社会民主党强调立法手段,布尔什维克强调先锋主义和最后一击。无政府主义者在原则上是反教条主义的,而蒲鲁东在本质上没有系统化甚至一致性的能力——因此许多不同的政治思想和运动都要归功于蒲鲁东,从主流的无政府主义者和工团主义者到各种革命的反资本主义君主主义者和法西斯主义者。(当然,法西斯主义可以让任何哲学适应它的目的——包括马克思主义,我们可以从俄罗斯的“红棕色联盟”和今天的“新右派”中看到这一点。)蒲鲁东本人在他后来关于联邦制的著作中,试图为这种放荡不羁调整他的理论,我们将会看到。但所有这些都发生在1844年之后,因此与我们的问题没有太大关系。也许把整个历时性的问题(可能发生了什么,或者确实发生了什么)放在一边,转而集中讨论马克思和蒲鲁东的共时性观点是有益的,这种观点不必局限于他们在1844年在时间和空间上的巧合,也不必局限于他们此后的分歧。毕竟,我们不是在寻求历史的评判。我们觉得自己处于绝望的境地;我们在寻求帮助。


在我看来,马克思和蒲鲁东之间的一项真正基本和重要的协议涉及财产或私有财产的性质。由于我们生活在一个超过 90\% 的财产除了作为货币之外不存在的时代,而且大约 430 人“控制”了比全球一半人口还多的货币,我们可能期望找到马克思和蒲鲁东有点过时了。但相反,如果可能的话,它们似乎比在(第一)19 世纪中叶表达时更新鲜。基于蒲鲁东的《什么是财产?马克思 1844 年的经济和哲学手稿将成为 1996 年对太晚变异的诺斯替资本的批判的令人钦佩的基础。这可能是因为 1848 年至 1991 年之间的“一切”(即整个社会运动)已被“历史的终结”一扫而空。在 1844 年可能胜利的资本现在终于在 1996 年真正胜利了。蒲鲁东和马克思所发现的资本的内在本质现在已经外化为真实的形式。例如,在媒体或生物工程技术中对模拟的欣喜若狂的实现,已经是马克思和蒲鲁东分析的异化所固有的。在他们后来的著作中,如果我们把那些将他们分开的段落放在一边,我们可以找到很多东西来补充我们的综合。我们不必局限于 1844 年之前写的文本。总的来说,我们会发现马克思对理解经济学和货币最有用,但对他关于权威和组织的思想的兴趣要少得多。至于蒲鲁东后来的互惠主义经济学说,我们可能有很多值得学习的地方——但总的来说,我们将对蒲鲁东关于权威和组织的思想更感兴趣——如果仅仅是因为我们知道马克思主义组织的走向,但我们确实知道不知道无政府主义组织可能会把我们带到哪里。这种研究所暗示的综合至少需要一整本书才能完成,而我们只能提供一些试探性的开端。


我们可能首先会问,资本在其普遍霸权的时刻,是否最终处于马克思经常预测的最终危机的边缘。根据定义,社会死亡的时刻就是社会重生的时刻;像凤凰一样,它从自己的灰烬中重生。但是最初是什么引起了这场大火呢?资本在五年内爆炸了,用热气体填充了一个巨大的南海气泡,膨胀直到它在一个脆弱的膜中包围了地球,像肥皂膜一样薄,一种封装了世界的货币天气。资本是“自由的”(例如作为迁徙或游牧资本),但同时资本是完全自我封闭的。苏联的废墟可能无法为其作为一个封闭系统的无限扩张提供新的市场资本需求。网络空间根本不是一个真正的“市场”,而仅仅是作为一个整体的资本的概念空间,以及它的所有表现形式。今天,股市仍在飙升,而在世界各地(甚至是前“第一世界”),资本只是抛弃它们,继续前进,就形成了枯竭区。其中一些区域不是地理上的,而是包括人口统计群体(例如无家可归者),或种族群体,或整个阶层的人。其他包括地理,甚至整个国家,如在非洲的某些地区。国际货币基金组织和世界银行(可能是我们最接近权力制度化的机构)除了实施惩罚性纪律和为制度化的新自由主义提供廉价劳动力之外,对拯救这些枯竭的地区无能为力。为了对抗共产主义,资本主义曾经不得不与各种潜在的盟友达成交易:例如,与劳动贵族的交易,或者与民主改革的交易,或者与有组织的宗教的交易。但是掌权的资本不再需要这些交易中的任何一个——至少它是这样认为的,并且现在正在背叛它所有的前盟友。世界上有没有不欠资本的人类?除了左派的残余,或“第三世界”中立主义的破碎碎片之外,还有值得认真对待的全球资本抵抗力量吗?(也许是伊斯兰教?)如果一颗巨大的流星离地球足够近,能够“擦除”地球上每一台电脑的磁带,会发生什么——所有的纯钱会怎么样?什么样的灾难会打破平衡,颠覆资本主义的想象力?一系列的波帕尔和切诺比尔?墨西哥或印度尼西亚的激进民粹主义起义?或者那个长期的最爱——生产过剩的危机,发展到令人厌恶的极端——破产,世界萧条?但是——单纯的生产(10\%)能在货币崩溃(90\%)后存活下来吗?通过对马克思的研究,这些问题中的一些也许可以找到答案。网络马克思。


蒲鲁东和马克思都讨论了异化。马克思更具哲学性的分析仍然比蒲鲁东的更有用,尽管他本人可能未能发展它。两位思想家都没有预见到媒体会在多大程度上加剧异化,在媒体中,越来越多的自主性从“日常生活”(主体和客体之间的一些直接关系仍然可能存在)中流失,并“转化为再现”语言和手势总是被事物,被“死资本”拦截。当民族国家和社区都沦为控制的景象时,充当资本的执行者或骗子,被剥夺了真正的权力,异化作为不属于我们的权力的真实和最有力的表现而出现,并且直接面对我们自己的枯竭和迟到。异化本身以权力为我们调解。这被称为工作问题,因为异化的劳动力是占位并在大多数时候为我们大多数人呈现贫困等级的力量。但我们也在消遣和淡化休闲中的真实性和所有由交换定义的关系以及“在工作中”进行“工作”——这个领域通过调解扩大到占据所有个人和社会空间,除了一些无意识和虚无主义的角落。调解原则上反对在场和分离,这解释了为什么它本身成为整体的原则。这种权力的形式依赖于分离(如等级、分工、异化等),同时依赖于同一性;全球文化完全类似于单一文化作为多样性认知贫困(misère)的消耗。受限于单一话语世界(中介)内的无限“选择”,主体性被“封闭”在单调和失范中,就像公共领域曾经代表单一文化和资本一样被封闭。随着对立的景观(苏联)的崩溃,在一个中介的分离话语中,相同的景观被夸大到全球比例(“淫秽”,“模拟”)。


在这个意义上,资本将实现单一的理性世界意识,这是启蒙运动的既定目标。这个目标是所有哲学家的继承者所共有的,包括民主主义者、资本家、马克思主义者和无政府主义者。从这个角度来看,整个殖民主义项目是合理的,甚至马克思自己也是如此;事实上,共产主义与资本主义一样,与这种“单一愿景”的出现有着密切的关系——除非不同的意识被相同的意识所取代,否则任何一种体系都无法“传播”。因此,这两种体系都存在固有的帝国主义。无政府主义,谴责分离的政治,然而呼吁同样单一的“科学”意识形式在理论和实践中实现自己。所有进步力量都同意,进步是随着意识的同质化而来的。差异被归入反应、迷信、偏见和无知的逐渐缩小的领域。


资本已经几乎达到了所有宗教的目标,因为金钱现在几乎完全是“精神的”,却包含并行使着世界上的所有权力。现在资本也将执行整个反宗教计划(宇宙祛魅),因为它将“进化”出一种统一的、无所不在的、扁平的、自我反省的、成功异化的意识——世界对自身的表征它本身是理性的、有启发性的、自由的——但不知何故完全是单向度的。现在,如果我们能暂时相信我们革命的凤凰,相信真正的反对派的再次出现,我们可能会问,旧左派的进步科学主义是否可以用来抵抗资本的进步科学主义——或者我们是否需要一个全新的意识概念。问题是要避免落入旧右翼及其反进步的反科学主义的范式,这种范式植根于神权封建主义,仅仅是出于我们对全球主义单一文化的厌恶反应。从某种意义上说,这个项目(可能被称为宇宙的再附魔)构成了过去 50 年的大量哲学、人类学和政治理论。在既反资本主义又反共产主义(或者可能是后资本主义和后共产主义)的认识论上投入了大量的努力。这些哲学是在“既不\Slash{}也不”或“第三条道路”的标题下看到的。但现在已经没有第三条路了,因为第一条和第二条都已经内爆到了超凡资本的同一个世界。例如,当没有第二世界甚至第一世界时,我们怎么还能说“第三世界”呢?第三种哲学在定义一种摆脱启蒙\Slash{}反启蒙二分法的意识方面取得了一些理论上的成功,从而摆脱了科学和反科学的威权主义。但第三条道路的政治假设是基于铁幕的“现实”和对立景象的霸权。现在形势发生了变化,基于第三路思维的政治分析必须调整以适应新的条件。左翼的旧“国际无产阶级团结”是建立在启蒙理性之上的,不亚于资本的国际资产阶级团结。但现在我们却被沉淀到一个理想统一的世界里,无论我们喜不喜欢,我们都“超越了左右”。全球资本既不是(或两者!)左也不是右——所以我们必须是\dots{}\dots{}别的东西。不是左右的另一种综合,而是可能同时利用激进和保守的观点(正如保罗古德曼称自己为“新石器时代的保守派”!)。


全球资本依赖于同一性和分离性的矛盾融合。如果我们要反对它,我们必须探索由这样一个悖论产生的矛盾——即差异和感觉(presence)。


在我们项目的这一部分,蒲鲁东可能比马克思更有用。尽管蒲鲁东和其他19世纪的进步人士一样对科学意识表示了敬意,但他的崇拜并不一致。他意识到不同的重要性;例如,他非常欣赏自己的地域差异,因为他是弗朗什-孔托伊(Franche-Comtois)的农民。与雅各宾派和共产主义者不同,他总是反对集权(因此他对官僚主义的谴责比马克思主义者更真诚),他同情农民和小资产阶级,也同情无产者(尽管他的最后一本书将工人阶级提升到了革命中的首要地位)。


在他后来的著作中,他甚至放弃了辩证法(在综合中达到顶点),转而支持基于矛盾的现实观。他认为矛盾是永恒的,它应该被协调和平衡,而不是被调和和消除。在此基础上,他能够将他的共同主义经济学与一个可以被称为无政府主义的政治体系结合起来,尽管他称之为简单的联邦主义(《联邦的原则》,1863年)。在这一体系中,群体的形成出于各种原因(经济、文化、地域等)。)可以加入一个基于经济和行政管理而非政治统一的联盟。从保留脱离权的意义上来说,每个群体都必须被视为自治的。[注:当苏联宪法保证苏联境内所有“自治”共和国的分离权时,解除了这种无政府联邦主义思想——这是一项从未被授予的权利。]


作为占有的财产继承了作为资本的财产,导致了大致的经济平等(有天赋和精力的空间,但生活有保障)。在群体内,所有与联盟关系无关的事情都留给群体去管理,甚至还有自主个体的潜在空间(因此诉诸个人主义无政府主义)。显然,一个人可以同时属于几个团体(劳工联合会、消费者合作社、邻里联盟、民兵组织等)。)—差异(“矛盾”)的可能性是无数的,存在的可能性也是无数的。无论是在政治上还是文化上,代议制帝国都被粉碎了。与马克思主义不同,这种社会联邦主义允许不止一种意识——真正的多元化而不是中介多元化。与资本主义不同,蒲鲁东的体系允许甚至要求存在(“团结”),因为中介关系不能满足联邦经济的迫切需要,更不用说包含联邦文化的非中介快乐了。


换言之,蒲鲁东的体系为我们提供了一种关于革命差异和革命感觉的理论——因此它应该为我们构成对资本的同一性和分离性的反对提供一些指导。 蒲鲁东的联邦制在无政府主义者古斯塔夫·兰道尔(Gustav Landauer)中得到了最彻底的发展,他试图在 1919 年的慕尼黑苏维埃中短暂地实施这些想法(并在 5 月 2 日苏联解体时被原纳粹分子杀害)。 特殊性对兰道尔来说是宝贵的,他不希望看到文化被资本主义或马克思主义同质化。 与众不同就是自由。 他设想了一个反对右翼的volkisch威权主义的Volk社会主义。 朗道尔虽然反宗教,但对“灵性”的理解已经足够,能够实现非平凡意识的现实和价值。 在这里,我想通过朗道尔的版本来看看蒲鲁东的联邦制,看看它在我们的项目中是否有用。


当代左派在特殊性问题上正经历着巨大的痛苦。例如,德勒兹和瓜塔里愿意在同性恋运动、儿童、精神病人、受压迫的少数民族等例子中考虑革命的“分子性”和抵抗的“异质性”。但是大多数左翼分子很难将这种开放的思想扩展到(比如)一个对萨满教和激进保护感兴趣的美国土著部落——或者更糟,扩展到一群对基督教感兴趣的贫穷白人。换句话说,只有当特殊性出现在定义19世纪进步工程的“科学理性主义”的整体意识中,或者被其取代时,大多数左翼分子才能接受特殊性。比如,大多数左翼分子完全无法将伊斯兰教视为原则上与资本主义的单一文化相对立,并因此成为一股潜在的革命力量,因为他们是在伊斯兰教是“非理性的”、“狂热的”和“落后的”的观点下成长起来的——这种观点将一切归功于启蒙沙文主义,而与共同的人性毫无关系。大多数左翼分子会支持萨帕塔主义者,因为他们代表了左翼思想进入后共产主义斗争世界的有效延续——但同样的左翼分子会表达自己的“担忧”,因为萨帕塔主义者是玛雅印第安人,他们想成为玛雅印第安人,而不是像“我们”这样的世俗社会主义光明会。也许“我们”应该开始学习如何行动,就好像我们真的相信不止一种意识(因此不止一种身份)可以在致力于“经验自由”的运动中蓬勃发展。我们应该停止吹嘘放弃我们以欧洲为中心的单向度的资产阶级世界观,而应该实际去做。


塞尔维亚民族主义和种族清洗呢⁈ 这不是报复性的特殊主义吗?


不,那是老式的民族沙文主义,表现为帝国主义的侵略,完全与资本串通一气,不可能与萨帕塔主义者、妇女、男女同性恋者或非裔美国人的反霸权主义混为一谈。


但是——既然资本成功地统一了意识,我们难道不应该庆幸它替我们做了工作,准备了马克思预言的革命吗?


不——因为即使统一的意识是一件好事,它也没有被资本获得,资本只对意识的同质化(或同质性)负责,这是对启蒙运动的恶意模仿,而不是它的真正实现。我们在这里攻击的不是理性本身。作为一种意识模式,我们甚至可以说它供应不足,我们可以更多地使用它,而不是更少。这是对理性主义的攻击。从历史上看,理性主义是作为“西方人”的霸权特质出现的,他们可能是左派或右派,但与法律之外的较小群体(妇女、儿童、“本地人”等)相比,他们总是正确的。).西方人将自己的异化视为自然对智人的计划——即使这有时会让他变得有点神经质——即使这有时会让他变得非常不理性——但这仍然是世界上唯一的真实意识。如果是这样,我会乞求(像波德莱尔的灵魂),“任何地方!任何地方,只要它离开这个世界!”但是我不相信。我相信革命性的差异,也相信革命性的存在。


这样看来,在组织的问题上,我们从马克思主义的集中模式中学到的东西比从兰道尔和蒲鲁东的分散模式中学到的要少,因为组织或实践是意识或理论的自然伴随物。事实上,蒲鲁东的体系与其说是作为一个“乌托邦”或对未来的计划,不如说是作为一个抵抗的模型,一个对现在的策略。斗争中的联邦制度相当于一种没有意识形态头脑的“统一战线”;但更重要的是,它将能够在资本主义世界秩序的“外壳”内构建一个不同的经济秩序。引用来自IWW序言,有可能蒲鲁东启发的无政府工团主义运动的一些思想可能再次具有一定的效用。(例如,在美国,劳工被“推回到”大约1880年,当然有很多东西需要重新学习,如破坏、总罢工等概念。)问题在于,工会或辛迪加只是构成“革命联邦主义”抵抗运动所需的无数组织形式中的一种。甚至工作本身(如妇女的生育和家务劳动或儿童的“教育”)也不能很容易地在工会模式下组织起来,更不用说“非工作”的重要领域了,如节日的生产、创造的乐趣或身份的自豪感。联邦概念对无政府主义者的吸引力在于,它使反对资本的每一个运动和趋势都有可能联合或合作,无论它们是否是“无政府主义的”,而这种联合的结构(通过其实现存在的组织)仍然是一种本质上的无政府主义结构。作为一名反意识形态主义者,无政府主义者并不特别在乎其他人是否想成为玛雅人、穆斯林或理性主义者,只要任何人在任何时候都可以自由脱离,并且每个人在任何时候都尊重个人自主和社会团结。在反对霸权主义的单一条件下,任何个人或团体都可以加入反资本主义联盟。“革命之后”毫无疑问,斗争将会继续,并开始将这些自由扩展到社会的最深处——但我们可以担心那一天的到来


目前,我们甚至还没有开始组织抵抗联合会。左翼的残余继续着“后现代主义”时期常见的“问题特许经营”——除非他们仍然致力于更古老的模式。由此产生的意识碎片(分离)进入了资本的“看不见的手”,或者更确切地说,进入了仍然构成资本对社会现实的建构的“多元主义”和“民主”的中介模拟。一切都在表象中耗尽——所有的抵抗都可以商品化为“抵抗”,所有的生命都可以被拜物化为“生活方式”。在这种接近完全原子化的情况下,单一文化的力量似乎是绝对的。既然没有什么能逃脱它,那么消费者幸福的唯一选择就表现为失败者的病态痛苦。当布什在海湾战争期间说美国“没有和平运动”时,他可能只是说没有运动。世界上十分之九的人(至少)被排除在资本狂喜之外,并被留在了这个无限重复的 19 世纪地狱中。但是没有动静。什么都没有发生。不仅“历史”已经从你我这样的人手中夺走了,我们可能会怀疑,无论多么“有钱有势”的人,是否还能创造历史。全球新自由主义声称它不是一种意识形态,这是真的——或者更确切地说,它是一种意识形态,它把所有的想法都交给了资本的机器、利润的“底线”、“自由” 市场,到了电脑的神谕。没有人能控制。权力仍然存在,但缺少控制。 432 位亿万富翁并不真正构成“统治阶级”。民族国家的存在只是为资本的劳动和企业福利提供纪律。“上帝死了——我自己感觉不太好”,如保险杠贴纸上所说。但上帝重生为金钱。我的复活会是什么?


显然,这种状态不会持续下去——这至少意味着我们占据了一个客观上具有过渡性的历史“点”。要么我们投降,要么我们把自己想象成反对派——不管怎样,我们必须走出绝望的深渊。如果我们选择重新想象的任务,那么我们必须尽可能地利用我们被委托的过去。如果科幻小说似乎破产了,那是因为未来已经被劫持到网络空间——而我们仍然停留在19世纪。马克思、蒲鲁东、傅立叶、尼采——这些都是我们同时代的人。启发了萨帕塔公报的玛雅长老们——他们也是我们的同时代人:远古时代的男男女女。过去是有趣的,因为它总是在变化,而未来(至少是资本世界的未来)是无聊的,因为它仍然是诞生的,死亡的,永不改变的。我们居住的这个过去是未知的,未被描绘的,未被探索的。我们的任务是一种穿越时间结构的迁移,寻找我们被剥夺的空间。或许从这种游牧式的漂流中会产生一个发现,一把钥匙,一条通向迷宫的线索——一条走出19世纪的路。


Saturnalia 1996


NYC









% begin final page

\clearpage



% new page for the colophon

\thispagestyle{empty}

\begin{center}

中文无治主义图书馆|中文無治主義圖書館





\bigskip
\includegraphics[width=0.25\textwidth]{logo-zh.pdf}
\bigskip

\end{center}

\strut

\vfill

\begin{center}


Hakim Bey

翻译\textbar{}马克思和蒲鲁东逃离十九世纪






\bigskip


https:\Slash{}\Slash{}zhuanlan.zhihu.com\Slash{}p\Slash{}540183754
   

       

       

       

       

       

       

   



\bigskip
\textbf{nightfall.buzz}


\end{center}

% end final page with colophon


% end closing pages



\end{document}


% No format ID passed.



